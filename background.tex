\subsection{Prototyping (!)}
In the manufacturing cycle, prototyping plays a crucial role. It assesses the fit, functionality,
and form of the parts being prototyped. This phase is important because it provides a clear idea of 
the final product's appearance and allows for the identification of unnecessary elements in the design. 
By addressing these issues during the prototyping stage, it saves time and money that would otherwise be 
spent making corrections during the production phase.
\newline \newline
Prototyping can be traced back to 1859, when a method known as photosculpture was invented by 
the French inventor Francois Willeme \cite{Lengua2017}. By surrounding an object with twenty-four 
cameras, each separated by 15 degrees, and taking photographs simultaneously, each photograph could 
be projected onto a screen. Using the images, a pantograph, and pieces of wood, 3D sculptures could 
be recreated from the photographs. This method of 3D prototyping enjoyed brief success but was eventually 
abandoned due to its labor-intensive nature.
\newline \newline
Further on, in 1981, Hideo Kodama, who worked at the Nagoya Municipal Industrial Research 
Institute in Japan, released a paper describing what is likely the first 3D-printed object 
in history \cite{Lengua2017}. The paper outlines a system called ``Automatic Method for Fabricating 
Cubic Shapes.'' This method involved using a photo-hardening polymer to fabricate solid models by 
stacking layers 2 mm thick on top of each other, thereby creating the first 3D-printed object. 
Similarly, Charles Hull developed stereolithography in 1986 \cite{SU20181}. This method involves 
solidifying a liquid material when it is exposed to UV light, based on cross-sections of a 3D model.
Unlike Hideo Kodama, Charles Hull patented his method and achieved greater academic and commercial 
success. However, Hideo Kodama was eventually awarded the Rank Prize, a prestigious award given to 
outstanding inventors, which he shared with Hull.
\newline \newline
For prototyping, 3D printing is a widely used tool across various industries, including aviation,
medical, automotive, etc. This technology is based on the concept of solidifying materials. 
Typically, the process begins with the creation of a 3D model in CAD or a similar system. 
These programs generate a file containing instructions for a rapid prototyping machine, usually 
a 3D printer, to create the 3D model in solid form \cite{sriharsha2018rapid}. There are several 
techniques for producing a prototype in the realm of 3D printing, including:
\begin{itemize}
    \item Stereo lithography
    \item Selective Laser Sintering
    \item Fused Deposition Modeling
\end{itemize}
While these technologies differ in their specific processes, they share similar advantages, such as 
reducing product development and manufacturing costs, thereby increasing competitiveness 
\cite{PHAM19981257}. Initially, these technologies and rapid prototyping were only accessible to 
large corporate firms, but today, 3D printers have become affordable for private individuals as well. 
Before the widespread availability of 3D printing, prototyping required skilled engineers to work from 
2D drawings.


\subsection{Virtual Reality (!)}
Virual reality  is a tool that artificialy simualtes our senses. While the technology might seem new, 
the first VR technology was introdoced by Ivan sutherland in 1968 
\subsection{Einride (!)}