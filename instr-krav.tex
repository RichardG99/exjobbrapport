För de olika funktionaliteterna (och/eller motsv) i ert system, hur ska ni avgöra om de är tillräckligt bra utförda/implementerade? Var går gränsen för ``tillräckligt bra''? (Eller när är de ``för dåliga''?)

Skilj på funktionalitet (vad ska systemet kunna göra) och krav (hur bra ska systemet vara). Själva funktionaliteterna har ni redan beskrivit i systemstrukturen eller huvuddelen nedan. (Har ni krav på saker ni beskriver först i huvuddelen kan ni lägga det här avsnittet efter huvuddelen.)

Skriv tydliga krav \emph{som går att utvärdera}.  (Hur snabbt? Hur många användare? Hur strömsnålt? eller vad som är relevant).

Beskriv hur utvärderingen ska gå till (automatiserade belastningstester, mätningar, en\-käter, fokusgrupper\ldots).
Beskriv hur externa intressenter involveras i utvärderingen.
