För de olika funktionaliteterna (och/eller motsv) i ert system, hur ska ni avgöra om de är tillräckligt bra utförda/implementerade? Var går gränsen för ``tillräckligt bra''? (Eller när är de ``för dåliga''?)

Koppla kraven till syfte/mål, eller beskriv varför det inte går/är rimligt att utvärdera.

Skilj på mål och krav. Beskriv här vilka krav ni har satt upp för ert system, och hur ni ska utvärdera om de är uppfyllda. Beskriv inte funktionalitet hos systemet eller mål/syfte: de beskrivs ju i andra delar av rapporten.

\emph{OBS:} Skilj på funktionalitet (vad ska systemet kunna göra) och krav (hur bra ska systemet vara). Själva funktionaliteterna har ni redan beskrivit i systemstrukturen eller huvuddelen nedan. (Har ni krav på saker ni beskriver först i huvuddelen kan ni lägga det här avsnittet efter huvuddelen.)

Att en funktionalitet överhuvud taget finns blir ett ``binärt'' krav som ni inte behöver utvärdera -- ni vet att ni gjort det eller inte och beskriver helt enkelt det i ett annat kapitel  -- men vilken funktionalitet har ni krav på \emph{hur bra} den ska vara (med olika tolkningar av ``bra'')? De kraven ska beskrivas här.

Skriv tydliga krav \emph{som går att utvärdera}.  (Hur snabbt? Hur många användare? Hur strömsnålt? eller vad som är relevant). Se till att kraven är väldefinierade (vad betyder ``snabbt'' eller ``effektivt''?)

Beskriv hur utvärderingen ska gå till (automatiserade belastningstester, mätningar, en\-käter, fokusgrupper\ldots).
Beskriv hur externa intressenter involveras i utvärderingen.

Om ni upptäcker att det är svårt att utvärdera något kanske det beror på att det inte är ett väldefinierat krav. Att t.ex. säga att ``systemet ska vara flexibelt och välskrivet'' är svårt att kolla, även om ni kanske provar med kodgranskning.
 Om ni formulerar det tydligare är det mindre svårt: vilka aspekter var det ni tittar på när ni gör kodgranskningen, har ni krav på de aspekterna? Beskriv isåfall dem, och beskriv kodgranskning som en metod (gärna med referens). Att systemet är ``modulärt och väldokumenterat'' kan förtydligas till att ni följer en viss kodstandard (beskriv då vilken, med referens).

Krav och utvärdering kan beskrivas tillsammans (räkna upp krav och hur det utvärderas) eller separat (först alla krav, sen hur de utvärderas). Det är viktigt att det är tydligt, och att ni inte har krav utan någon utvärderingsmetod.

%%% Local Variables:
%%% mode: latex
%%% TeX-master: "rapport-mall"
%%% End:
