Här beskriver ni i princip er problemformulering.  I detta avsnitt ska framgå syfte, mål, och motivation med projektet. 
Dessa behöver dock \emph{inte} vara separata underrubriker.

\paragraph{Syfte.} Vart strävar projektet? vad är det övergripande målet, nyttan, effekterna av projektet?  (t.ex. bättre koll på kosthållning, enklare planering av studier\ldots)
\paragraph{Mål.} Vad ska konkret levereras/utföras av projektet, för att ta oss närmare syftet?
\paragraph{Motivation.}  Varför är projektet viktigt?  Vilka är det viktigt för, vilka externa intressenter finns?  Hur stort är problemet, vad är följden av att det inte är löst, hur bra vore det att lösa?  Vilken ``lucka'' i området täcker ni?
Varför är er lösning bättre/annorlunda än andras?

Se till att ni i detta avsnitt övertygar läsaren om att problemet finns, att det inte är löst, och att det är viktigt att lösa. Ju starkare argumentation och motivation (med källor) dess bättre.
\begin{itemize}
\item Visa att det finns ett problem.
\item Visa att problemet är viktigt att lösa, att det behöver lösas.
\item Visa att problemet inte redan är löst.
\end{itemize}

I det här avsnittet kan ni också börja beskriva etiska aspekter och hållbarhetsaspekter, men det finns förstås flera naturliga ställen att ta upp dem (\emph{till exempel} sektionerna~\ref{sec:metod}, \ref{sec:krav}, \ref{sec:resultat} och~\ref{sec:slutsatser}, men kanske redan i sektion~\ref{sec:introduktion} eller \ref{sec:bakgrund}).

Det är helt OK (och bra!) att också beskriva negativa/kritiska aspekter av ert projekt och arbete, inte bara positiva/goda. 

\emph{Använd kursmaterialet} för att få stöd att utveckla etiska och hållbarhetsaspekter.
Tänk t.ex.~på stödfrågorna för att tänka på etiska aspekter:
\begin{itemize}
\item Vilka är \emph{direkta} intressenter och hur påverkas de? (användare, företag, kunder)
\begin{itemize}
\item  Vad krävs för att kunna använda er lösning? (kunskap, förmågor, resurser)
\item  Vilka exkluderar ni?
\item  Vad underlättar ni och vad gör ni svårare?
\item  I vilka sammanhang kan er lösning användas och inte användas?
\end{itemize}
\item  Vilka är \emph{indirekta} intressenter och hur påverkas de? (familj, samhälle, konkurrenter)
\item  Kan tekniken användas för ``fel'' syften?
\item  Hur ser ett samhälle ut där er lösning används i stor skala?
\end{itemize}

För hållbarhetsfrågor, prova gärna ``Futures Thinking'' och ``Systems Thinking'' (se länkar i kursmaterialet).
