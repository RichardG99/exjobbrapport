% MUST use a4paper option
% MAY use twoside, smaller font, and other class - but not for Självständigt arbete i IT
\documentclass[a4paper,12pt]{article}
% Use UTF-8 encoding in input files
\usepackage[utf8]{inputenc}
% Use T1 font encoding to make the \hyphenation command work with UTF-8
\usepackage[T1]{fontenc}
\usepackage{dirtytalk}


% Om ni skriver på svenska, använd denna rad:
\usepackage[english,swedish]{babel}
% If you are writing in English, use the following line INSTEAD of the previous (note order of parameters):
% \usepackage[swedish,english]{babel}

% Use the template for thesis reports
\usepackage{UppsalaExjobb}

% För att göra ett index behövs
%  - \usepackage{makeidx}
%  - \makeindex i "preamble", dvs före \begin{document}
%  - \printindex, typiskt sist, före \end{document}
% - och att man lägger in \index{ord} på olika ställen i dokumentet
\usepackage{makeidx}
\makeindex


% Designval: per default används styckesindrag, men ibland blir det snyggare/mer lättläst med tomrad mellan stycken. Det åstadkoms av de följande raderna.
% Tycker ni om styckesindrag mera, kommentera bort nästa två rader.
\parskip=0.8em
\parindent=0mm

% Designval: vill ni ha en box runt figurer istället för strecken som är default, av-kommentera raden nedan. Obs att både \floatstyle och \restylefloat behövs.
%\floatstyle{boxed} \restylefloat{figure}

\begin{document}
% För att ställa in parametrar till IEEEtranS/IEEEtranSA behöver detta ligga här (före första \cite).
% Se se IEEEtran/bibtex/IEEEtran_bst_HOWTO.pdf, avsnitt VII, eller sista biten av IEEEtran/bibtex/IEEEexample.bib.
%%%% OBS: här ställer ni t.ex. in hur URLer ska beskrivas.
\bstctlcite{rapport:BSTcontrol}

% Set title, and subtitle if you have one
\title{Rapportmall för självständigt arbete} % och uppsatsmetodik
% Use this if you have a subtitle
%\subtitle{beskrivande men gärna lockande}
\subtitle{version vinter 2022}

% Set author names, separated by "\\ " (don't forget the space, or use newline)
% List authors alphabetically by LAST NAME (unless someone did significantly more/less, which should not be the case)
% For drafts, include your email addresses to make it easier to send peer reviews
\author{Sofia Cassel \\ Björn Victor (bjorn.victor@it.uu.se)}

% Visa datum på svenska på förstasidan, även om ni skriver på engelska!
\date{\begin{otherlanguage}{swedish}  %\foreignlanguage doesn't seem to affect \today?
\today
\end{otherlanguage}}

% Använd detta om året för rapporten inte är innevarande år
%\year=2018

% Ange handledare, ämnesgranskare, examinator om dessa finns
% Extern handledare: t.ex på företag ni arbetat med?
\exthandledare{Hand Ledare, Firma Ment AB}
% Intern(a) handledare, i bokstavsordning på efternamnet
\handledare{NN och Björn Victor}

% This creates the title page
\maketitle

% Change to frontmatter style (e.g. roman page numbers)
\frontmatter

%%%% OBS: Läs också källkoden till alla instr-X.tex.
%%%% Tips: ni kan använda separata filer för de olika delarna i er rapport på motsvarande sätt,
%%%% men använd inte samma filnamn!

\begin{abstract}
\input{instr-abstract}
\end{abstract}

\begin{sammanfattning}
Sammanfattning, alltid på svenska. Se till att det står samma saker i den svenska sammanfattningen och det engelska abstractet.

\textbf{Tips:} skriv först på det ena språket (dvs abstract \emph{eller} sammanfattning), och när den versionen är klar, översätt den då till det andra språket. Lämna ``det andra språket'' tomt eller med en notering om att den skrivs när ``det ena språket'' är klart.

\begin{enumerate}
\item Vad är problemet, ämnet?
\item Hur angreps/löstes problemet?
\item Vad är resultaten, hur väl löstes problemet?
\item Hur bra blev resultaten, hur användbara är de?
\end{enumerate}

Ca 10-20 rader. Använd inte referenser; ej heller formler om det går att undvika.

Sammanfattningen ska vara förståelig utan att läsa resten av rapporten, och resten av rapporten ska kunna läsas utan att läsa sammanfattningen. Det är helt OK att återanvända text från introduktionen.

%%% Local Variables:
%%% mode: latex
%%% TeX-master: "rapport-mall"
%%% End:

\end{sammanfattning}

% Innehållsförteckningen här.
\tableofcontents

% Här kan man också ha \listoffigures, \listoftables

\cleardoublepage

%%%%%%%%%%%%%%%% Ta bort allt mellan här och \mainmatter (inkl \newpage) (men inte \mainmatter) i slutversionen
\section*{Hur ni använder detta malldokument}
Titta i källdokumentet för diverse inställningar för författare, titel, etc. Läs också käll\-doku\-men\-ten för instruktionerna i filerna \verb|instr-X.tex| för olika värden på \verb|X|.

\emph{OBSERVERA} att de ``fasta fält'' som blir på svenska (trots att ni ställt in engelska med \texttt{babel}), som Examinator, Handledare, datum på framsidan osv, \emph{ska} vara på svenska oavsett språk i rapporten. Abstract ska alltid vara på engelska, medan Sammanfattning alltid ska vara på svenska.

I flera appendix finns mer info och instruktioner som inte gäller själva rapportstrukturen.

För att slippa få med instruktionerna för rapportstrukturen i era inlämningar, ta bort \verb|\input{instr-X}| för alla värden av \verb|X|
i källdokumentet.

\textbf{Tips:} ni kan använda separata filer för de olika delarna i er rapport på motsvarande sätt, men använd inte samma filnamn!

\subsection*{Generellt}
Varje numrerat avsnitt ska finnas med i er slutrapport, om inget annat anges.  
Välj rubrik på svenska eller engelska beroende på ert valda rapportspråk.

Om ni skriver på engelska ska titeln skrivas med första bokstaven i varje ord versal, utom ``småord''. Exempel: \emph{A Really Interesting Project on the Fundamentals of Shoes}\footnote{Se t.ex.~\url{https://en.wikipedia.org/wiki/Capitalization\#Titles}}.  (Detta gäller även titlar i referenser på engelska.)
Rubrikerna i texten kan skrivas på detta sätt eller som på svenska (stor första bokstav i meningen), men \emph{var konsekventa}.

Glöm inte att läsa kurslitteraturen~\cite{dawson:projects-in-computing,dawson:projects-in-computing-old}.

Undvik helst att lägga in en ordlista/glossary i början av rapporten. När man läser den kommer begreppen utan kontext och det kan vara svårt att förstå. Ni måste ändå förklara begreppen när de behöver förklaras (se appendix \ref{app:definiera-innan-anv}, sid~\pageref{app:definiera-innan-anv}).
% \subsection*{Uppdateringar av detta dokument}
% \begin{description}
% \item[2016-05-16]\mbox{}\\

% \end{description}

\newpage
\section*{Att göra}
En onumrerad sektion som beskriver läget för rapporten kan vara användbart i ``veckans inlämning'' för att underlätta feedbacken.

För att hantera ``att-göra-listor'' i rapporten kan La\TeX-paketet \verb|todonotes| kanske vara användbart, både för att själva komma ihåg vad som ska ändras/läggas till, och för att visa det för läsarna under skrivandets gång. Se \url{http://ctan.org/pkg/todonotes} för mer info.

Naturligtvis är det generellt bra att också använda ett projekthanteringssystem som Trello\footnote{\url{https://trello.com/}} el.dyl. för att hålla reda på allt som är kvar att göra!

%%% Local Variables:
%%% mode: latex
%%% TeX-master: "rapport-mall"
%%% End:

\newpage
%%%%%%%%%%%%%%%% OBS! Ta bort allt mellan \mainmatter och här (inkl \newpage) i slutversionen

% Change to main matter style (arabic page numbers, reset page numbers)
\mainmatter

% Here comes the text of the report.

\section{Introduktion eller Inledning / Introduction}
\label{sec:introduktion}
Beskriv åtminstone samma saker som i abstract, men mer utförligt -- typiskt 1-2 sidor. Spara tekniska detaljer till senare, eftersom läsaren inte är insatt än.

\begin{enumerate}
\item Vilket är området ni arbetar inom? Vad är problemet, ämnet, kontexten? 
\item Varför är problemet viktigt/intressant att lösa?
\item Hur angreps/löstes problemet? 
\item Vad är resultaten, hur väl löstes problemet?
\item Hur bra blev resultaten, hur användbara är de?
\end{enumerate}

Tänk på att börja introduktionen med en mening eller ännu hellre ett helt stycke som ``fångar'' läsaren och motiverar läsaren att fortsätta läsa.  \emph{Vi har valt att göra ett projekt om X} är relevant för er, men kommer inte att vilja få någon att läsa vidare.  Försök åtminstone få med någon slags bakgrund/kontext och (helst) motivation att fortsätta läsa.  Typ \emph{X är ett programspråk som tagit världen med storm.  Vi vill utforska om man kan kombinera X med Y för att göra\ldots}

Se till att ni \emph{kommer till kritan snabbt} – man vill inte läsa igenom två stycken text innan man får veta vad ni tänker göra i ert projekt.  Börja t.ex. \emph{inte} med att presentera alla idéer ni inte valt – läsaren vill veta vad ni ska göra, inte vad ni inte ska göra. 
Använd gärna en bild som visar vad det är ni åstadkommit.

Översiktlig beskrivning av systemet och dess features ska vara under systemdesign / systemstruktur, inte i introduktionen.

Introduktionen bör vara begriplig för t.ex.~en student i årskursen under, och gärna för en ännu bredare läsarkrets.

Avsluta gärna med en överblick över hela rapporten, där ni \emph{kortfattat} beskriver vad de olika kapitlen handlar om.

%%% Local Variables:
%%% mode: latex
%%% TeX-master: "rapport-mall"
%%% End:


\paragraph{Tillkännagivande eller Tack / Acknowledgement}
\input{instr-acknowledgement}

\paragraph{Redovisning av arbetsfördelning / Declaration of division of labor}
\input{instr-arbete}


\section{Bakgrund / Background}
\label{sec:bakgrund}
\input{instr-bakgrund}

\section{Background (!)}
\subsection{Prototyping (!)}
In the manufacturing cycle, prototyping plays a crucial role. It assesses the fit, functionality,
and form of the parts being prototyped. This phase is important because it provides a clear idea of 
the final product's appearance and allows for the identification of unnecessary elements in the design. 
By addressing these issues during the prototyping stage, it saves time and money that would otherwise be 
spent making corrections during the production phase.
\newline \newline
Prototyping can be traced back to 1859, when a method known as photosculpture was invented by 
the French inventor Francois Willeme \cite{Lengua2017}. By surrounding an object with twenty-four 
cameras, each separated by 15 degrees, and taking photographs simultaneously, each photograph could 
be projected onto a screen. Using the images, a pantograph, and pieces of wood, 3D sculptures could 
be recreated from the photographs. This method of 3D prototyping enjoyed brief success but was eventually 
abandoned due to its labor-intensive nature.
\newline \newline
Further on, in 1981, Hideo Kodama, who worked at the Nagoya Municipal Industrial Research 
Institute in Japan, released a paper describing what is likely the first 3D-printed object 
in history \cite{Lengua2017}. The paper outlines a system called ``Automatic Method for Fabricating 
Cubic Shapes.'' This method involved using a photo-hardening polymer to fabricate solid models by 
stacking layers 2 mm thick on top of each other, thereby creating the first 3D-printed object. 
Similarly, Charles Hull developed stereolithography in 1986 \cite{SU20181}. This method involves 
solidifying a liquid material when it is exposed to UV light, based on cross-sections of a 3D model.
Unlike Hideo Kodama, Charles Hull patented his method and achieved greater academic and commercial 
success. However, Hideo Kodama was eventually awarded the Rank Prize, a prestigious award given to 
outstanding inventors, which he shared with Hull.
\newline \newline
For prototyping, 3D printing is a widely used tool across various industries, including aviation,
medical, automotive, etc. This technology is based on the concept of solidifying materials. 
Typically, the process begins with the creation of a 3D model in CAD or a similar system. 
These programs generate a file containing instructions for a rapid prototyping machine, usually 
a 3D printer, to create the 3D model in solid form \cite{sriharsha2018rapid}. There are several 
techniques for producing a prototype in the realm of 3D printing, including:
\begin{itemize}
    \item Stereo lithography
    \item Selective Laser Sintering
    \item Fused Deposition Modeling
\end{itemize}
While these technologies differ in their specific processes, they share similar advantages, such as 
reducing product development and manufacturing costs, thereby increasing competitiveness 
\cite{PHAM19981257}. Initially, these technologies and rapid prototyping were only accessible to 
large corporate firms, but today, 3D printers have become affordable for private individuals as well. 
Before the widespread availability of 3D printing, prototyping required skilled engineers to work from 
2D drawings.


\subsection{Virtual Reality (!)}
Virual reality  is a tool that artificialy simualtes our senses. While the technology might seem new, 
the first VR technology was introdoced by Ivan sutherland in 1968 \cite{lavalle2023virtual}. Today the VR
industry is worth billions of dollars and is found in many industries
\subsection{Einride (!)}

\section{Syfte, mål, och motivation / Purpose, aims, and motivation}\label{sec:syfte}
\input{instr-syfte}

\subsection{Avgränsningar / Delimitations}
\input{instr-avgransningar}

\section{Relaterat arbete / Related work}
\input{instr-relaterat}

\section{Teori / Theory}
\label{sec:teori}
\input{instr-teori}

\section{Metod/Tillvägagångssätt/Tekniker eller Method/Approach/Techniques}
\label{sec:metod}
Här beskriver ni vilka metoder/verktyg/tekniker/approacher ni använt för att lösa problemet / besvara frågeställningen.  Vilka metoder har ni konkret använt för att lösa problemet/bygga systemet?  Vilka tekniker/verktyg använde ni?

Observera att det inte är samma sak som att beskriva \emph{hur} ni använde teknikerna/verktygen: det kommer i Del X, implementationsdelen (se avsnitt~\ref{sec:del-x}).

Glöm inte att \emph{motivera} era val av metoder. Finns det flera rimliga alternativ? Beskriv varför ni inte valt dem (t.ex.~varför er valda metod är bättre).
Visa att det är rimligt att använda just detta tillvägagångssätt.
Det gäller även i det fall det är givet på förhand vilken teknik ni ska använda (t.ex. vilket programspråk) för att det ska passa i ett sammanhang eller existerande system (t.ex. ett som ny bygger vidare på). 
Även om man har en given teknik kan man alltså behöva förklara att en annan egentligen vore bättre -- men att omständigheter gör att ni ändå måste välja den givna tekniken.

Det är ofta bra att börja med att förklara vad ni valt för teknik/verktyg, och därefter motivation och alternativ. Om man börjar med alla alternativ och väntar med att förklara vad man valt till sist, blir det inte lika enkelt att läsa.

Detta avsnitt ska \emph{inte} innehålla information om hur gruppen organiserat arbetet (github, trello, jira\ldots) \emph{om} det inte är relevant för resultatet (och det är det oftast inte).

Använd tydliga underrubriker, t.ex. ``Ramverk för webbplatsen'' snarare än ``Wordpress'', eller ``Databashanterare'' snarare än ``MongoDB''.

%%% Local Variables:
%%% mode: latex
%%% TeX-master: "rapport-mall"
%%% End:


\section{Systemstruktur / System structure}
\label{sec:systemstruktur}
\input{instr-systemstruktur}

\section{Krav och utvärderingsmetoder / Requirements and evaluation methods}
\label{sec:krav}
För de olika funktionaliteterna (och/eller motsv) i ert system, hur ska ni avgöra om de är tillräckligt bra utförda/implementerade? Var går gränsen för ``tillräckligt bra''? (Eller när är de ``för dåliga''?)

Koppla kraven till syfte/mål, eller beskriv varför det inte går/är rimligt att utvärdera.

Skilj på mål och krav. Beskriv här vilka krav ni har satt upp för ert system, och hur ni ska utvärdera om de är uppfyllda. Beskriv \emph{inte} funktionalitet hos systemet eller mål/syfte: de beskrivs ju i andra delar av rapporten.

\emph{OBS:} Skilj på funktionalitet (vad ska systemet kunna göra) och krav (hur bra ska systemet vara). Själva funktionaliteterna har ni redan beskrivit i systemstrukturen eller huvuddelen nedan. (Har ni krav på saker ni beskriver först i huvuddelen kan ni lägga det här avsnittet efter huvuddelen.)

Att en funktionalitet överhuvud taget finns blir ett ``binärt'' krav som ni inte behöver utvärdera -- ni vet att ni gjort det eller inte och beskriver helt enkelt det i ett annat kapitel  -- men vilken funktionalitet har ni krav på \emph{hur bra} den ska vara (med olika tolkningar av ``bra'')? De kraven ska beskrivas här.

Skriv tydliga krav \emph{som går att utvärdera}.  (Hur snabbt? Hur många användare? Hur strömsnålt? eller vad som är relevant). Se till att kraven är väldefinierade (vad betyder ``snabbt'' eller ``effektivt''?)

Beskriv hur utvärderingen ska gå till (automatiserade belastningstester, mätningar, en\-käter, fokusgrupper\ldots).
Beskriv hur externa intressenter involveras i utvärderingen.

Om ni upptäcker att det är svårt att utvärdera något kanske det beror på att det inte är ett väldefinierat krav. Att t.ex. säga att ``systemet ska vara flexibelt och välskrivet'' är svårt att kolla, även om ni kanske provar med kodgranskning.
 Om ni formulerar det tydligare är det mindre svårt: vilka aspekter var det ni tittar på när ni gör kodgranskningen, har ni krav på de aspekterna? Beskriv isåfall dem, och beskriv kodgranskning som en metod (gärna med referens). Att systemet är ``modulärt och väldokumenterat'' kan förtydligas till att ni följer en viss kodstandard (beskriv då vilken, med referens).

Krav och utvärdering kan beskrivas tillsammans (räkna upp krav och hur det utvärderas) eller separat (först alla krav, sen hur de utvärderas). Det är viktigt att det är tydligt, och att ni inte har krav utan någon utvärderingsmetod.

%%% Local Variables:
%%% mode: latex
%%% TeX-master: "rapport-mall"
%%% End:


\section{DEL x: Implementation av XYZ}
\label{sec:del-x}
\input{instr-del-x}

\section{Utvärderingsresultat / Evaluation results}
\input{instr-utv-resultat}

\section{Resultat och diskussion / Results and discussion}
\label{sec:resultat}
\input{instr-resultat}

\section{Framtida arbete / Future work}
\label{sec:framtida}
\input{instr-framtida-arbete}

\section{Slutsatser / Conclusions}
\label{sec:slutsatser}
\input{instr-slutsatser}

%%%% Referenser - SE OCKÅ APPENDIX

% Use one of these:
%   IEEEtranS gives numbered references like [42] sorted by author,
%   IEEEtranSA gives ``alpha''-style references like [Lam81] (also sorted by author)
%\bibliographystyle{IEEEtranS}
\bibliographystyle{IEEEtranSA}

% Here comes the bibliography/references.
% För att göra inställningar för IEEEtranS/SA kan man använda ett speciellt bibtex-entry @IEEEtranBSTCTL,
% se IEEEtran/bibtex/IEEEtran_bst_HOWTO.pdf, avsnitt VII, eller sista biten av IEEEtran/bibtex/IEEEexample.bib.
\bibliography{bibconfig,refs}
%\bibliography{refs}

\newpage
\appendix %%%% markerar att resten är appendixar
%%%% I er egen version, ta bort allt nedan (utom \end{document})
%%%% Här finns både instruktioner och tips - läs hela!

\section{Hur man gör appendix}
\label{app:appendix}
Appendixar kan vara bra för bilagor som enkätundersökningar, större kodavsnitt, etc. 

Appendix läggs efter referenslistan, och ska börja på en ny sida. Använd \verb|\newpage| för att göra ett sidbrott där resten av nuvarande sida är tom. Skriv sen \verb|\appendix| för att markera att resten är appendix, och 
 använd sen vanliga \verb|\section{}| för varje appendix, som kommer att ``numreras'' A, B, C osv.

För att referera till ett appendix, gör likadant som till ett avsnitt (se instruktion \vpageref[nedan]{sec:referera-labels}), till exempel: ``se appendix~\ref{app:latex} för tips om La\TeX-använding (see Appendix~\ref{app:latex} for tips about La\TeX{} usage).''

\section{Några tips för La\TeX-användning}
\label{app:latex}

Ett enkelt sätt att använda/\textbf{installera} LaTeX för MacOS är TexShop\footnote{\url{http://pages.uoregon.edu/koch/texshop}}.

För \textbf{samarbete} när man skriver La\TeX-dokument använder somliga Overleaf (\url{https://www.overleaf.com/}, tips på liknande system är välkomna), men det funkar också att använda git och vanliga texteditorer (t.ex Emacs). I det fallet är det smart att dela upp dokumentet i flera (t.ex. ett per kapitel) som sen inkluderas med \verb|\input{kapitel}|. Ett tips är också att slå på radbrytning i texteditorn, så att konflikter vid incheckningar hanteras per kort rad istället för per jättelång rad.

\textbf{Läs också i Wikibooks} (\url{http://en.wikibooks.org/wiki/LaTeX}), \textbf{missa inte} Appendix om ``Sample LaTeX documents'' (men använd alltid rapportmallen som bas).

\textbf{Citat-tecken} skriver man med \verb|``foo''| (dvs två bakåtfnuttar före, och två vanliga fnuttar efter). LaTeX gör så att det blir snyggt: ``foo''.

När man skriver på svenska behöver man ibland ``visa'' var ord (speciellt såna med med åäö) kan \textbf{avstavas} genom att använda \verb|\-| (liknande \textit{soft hyphen}): ämnesöversiktsintroduktion avstavas med några sådana instuckna på rätt ställen istället som ämnes\-över\-sikts\-intro\-duk\-tion

\begin{verbatim}
ämnes\-över\-sikts\-intro\-duk\-tion
\end{verbatim}

För att formattera \textbf{URLer} bättre (så att t.ex. radbrytning blir snyggare), skriv t.ex. \verb|\url{http://www.it.uu.se/research/group/concurrency}| i texten eller referensen.

\label{sec:referera-labels}
För att \textbf{referera} till avsnitt, figurer, tabeller etc, använd \verb|\label{markör}| för att ``sätta ett märke'' i text eller figur, och \verb|\ref{markör}| för att referera till den, t.ex. Läs mer \vpageref{figurers-namn} om hur man benämner avsnitt, figurer osv.
\begin{verbatim}
\section{Motivation}
\label{sec:motivation}
\end{verbatim}

följt av
\begin{verbatim}
Som vi nämnt i avsnitt~\ref{sec.motivation}...
\end{verbatim}
eller på engelska (notera ``Section'' med stor inledande bokstav)
\begin{verbatim}
As already mentioned in Section~\ref{sec.motivation}...
\end{verbatim}
Se också appendix~\ref{app:appendix} för hur man gör med appendix.


För att få referenser att inte hamna först efter ett \textbf{radbrott}, använd icke-brytande space \verb|såhär~\cite{fin-bok}|, där tilde-tecknet \verb|~| alltså gör ett obrytbart space. Detta är i princip också alltid rätt att använda före siffror (och i stora tal på engelska, t.ex. \verb|100~000| för 100~000), och förstås också före \verb|\ref{fig}|.

Använd \emph{aldrig} dubbel-backslash \verb|\\| för att få avbrott mellan stycken. Använd alltid dubbel ny rad för detta. Använd bara \verb|\\| i tabeller o.dyl.

För att göra ett \textbf{sidbrott} där resten av sidan blir tom, använd \verb|\newpage|, använd inte \verb|\pagebreak|. Det senare är till för att finjustera var latex gör ett automatiskt sidbrott, inte för att avsluta en halvfull sida.

Använd aldrig \textit{math mode} (dvs \verb|$Text$|) för att få kursiv text eller för flerbokstavs-variabler i matematiska uttryck, eftersom text i math mode tolkas som en multiplikation av de olika bokstäverna och då får konstiga mellanrum -- det blir särskilt tydligt där La\TeX{} normalt skulle ha gjort ligaturer (som fi). Använd istället \verb|\textit{Text}| (\textit{Text}) eller \verb|\textsl{Text}| (\textsl{Text}), eller kanske \verb|\textrm{Text}| (\textrm{Text}, speciellt i matematiska uttryck) eller \verb|\textsf{Text}| (\textsf{Text}, speciellt för kod).

\subsection{Bib\TeX-tips}

För att hantera bibliografi (\textbf{referenser}) på ett smidigt sätt, använd BibTeX! Läs mer i \url{http://en.wikibooks.org/wiki/LaTeX/Bibliography_Management#BibTeX}, nedan om referenser, och
\texttt{IEEEtran\_bst\_HOWTO.pdf} i directoryt \texttt{IEEEtran/bibtex} i detta Github-repository.

För att se till att BibTeX inte gör namn, förkortningar etc till lowercase, använd \verb|{}| och skriv typ:
\begin{verbatim}
title = {The {DSP} of {N}ewton applied to {iOS}}
\end{verbatim}

Skriv alltid månader för publikation med de inbyggda förkortningarna, typ:
\begin{verbatim}
month = jun
\end{verbatim}
istället för \verb|{jun}| eller \verb|"jun"| eller \verb|"June"| eller \verb|"Juni"|. Då kan nämligen bibliographystyle styra hur det förkortas etc.

Kom ihåg att separera författarnamn med ``and'', inte med komma. Ibland behöver Bib\TeX{} extra hjälp att förstå vad som är för- och efternamn, och \emph{då} är komma rätt att använda. Exempel (se~\cite{whitehead.russel:principia-mathematica} i referenslistan):
\begin{verbatim}
{Whitehead, Alfred North and Bertrand Russel}
\end{verbatim}

Ett verktyg för att hantera BibTeX-filer i MacOS är BibDesk\footnote{\url{http://bibdesk.sourceforge.net/}}.

\section{Referenser}
\label{sec:referenser}

Se också kap 8.5 i Dawson~\cite{dawson:projects-in-computing}.

\begin{center}
\fbox{\parbox{30em}{
\textbf{OBS: viktigt!}
Det finns åtminstone tre syften med utformningen av referenserna och referenslistan.
\begin{enumerate}
\item Man ska hitta referensen (från texten) i referenslistan.
\item Man ska förstå vad som refereras (vilken typ av referens det är) så att man kan värdera den.
\item Man ska kunna hitta referensen i verkligheten.
\end{enumerate}
Eftersträva alltid att uppnå alla tre.
}}
\end{center}

Använd numeriska referenser (IEEE-stil~[42]) eller nyckelordsbaserad~[Lam86], inte fotnotstil. Referenserna sorteras alfabetiskt efter författare/motsv i referenslistan. I LaTeX, använd \verb|\bibliographystyle{IEEEtranS}| eller \verb|{IEEEtranSA}| (eller liknande), se rapportmallen. \textbf{Börja} med att använda \verb|{IEEEtranSA}| som tydligare visar när viss info saknas i bibtex-entries (t.ex. år och författare).

För att göra inställningar för \verb|\bibliographystyle{IEEEtranS/SA}| kan man använda ett speciellt bibtex-entry som heter \texttt{@IEEEtranBSTCTL}, se mer info i filen \texttt{IEEEtran\_bst\_HOWTO.pdf} i directoryt \texttt{IEEEtran/bibtex}, avsnitt VII, eller sista biten av \texttt{IEEEexample.bib} i samma directory.

Referenserna skrivs i direkt anknytning till det som föranleder referensen (t.ex. ett påstående eller resultat), före eventuellt skiljetecken, och med ett fast mellanslag till föregående ord. I La\TeX, \verb|skriv~\cite{lam86}| för att få en ``non-breaking space''. Se också rapportmallen, och sista stycket på sid 211 i Dawson~\cite{dawson:projects-in-computing}. 

Man ska alltså \emph{inte} skriva referenserna efter ett längre stycke (som vissa verkar lära sig att göra, någonstans). Det gör det oftast otydligt vad som egentligen är hämtat från, eller styrks, av referenserna. I vissa fall kan man vilka göra en kort sammanfattning av vad en författare skriver i en artikel el.dyl., men att bara lägga på en referens sist i stycket är inte tillräckligt tydligt. Det är mycket bättre och tydligare att inleda stycket med att skriva något i stil med ``Lisa Lagom beskriver\verb|~\cite{lagom-bok}| hur X beror av Y och i sin analys visar hon i detalj hur sambandet ser ut\ldots''.

Att upprepa samma referens ofta i ett stycke (kanske efter varje mening) gör det mer svårläst. Försök att skriva om stycket så att det blir tydligt att det hela bygger på referensen, som lämpligen används tidigt. Exempel: ``I en undersökning av WHO\verb|~\ref{who}| beskrivs följderna av XYZ och de indirekta risker som uppkommer'', och därefter kan de olika följderna och riskerna tas upp i samma stycke utan att det blir otydligt.

När man refererar till ``tjocka'' saker som böcker är det lämpligt att ange sidnummer 
(som \verb|\cite[sid 211-214]{dawson:projects-in-computing}|) som blir \cite[sid 211-214]{dawson:projects-in-computing}, eller kapitel (som \verb|\cite[kapitel 5]{...}|), men för ``tunnare'' saker behöver man bara göra det för att speciellt peka ut om man t.ex. menar en viss del av referensen (kanske den tar upp tre olika sätt att göra X och man vill peka på det 3:e, inte de första två).

Upprepa inte samma referens (bok, artikel etc) i referenslistan för olika avsnitt, kapitel etc i referensen, utan gör på det sätt som beskrivs i stycket ovan.

Man skriver normalt inte ut titeln på det man refererar om det inte är ett speciellt viktigt verk, typ \emph{Principia Mathematica}~\cite{whitehead.russel:principia-mathematica} eller \emph{On Computable Numbers}~\cite{turing:computable-numbers}. Skriv alltså inte ``I artikeln \emph{How to increase efficiency in databases}~[Ref21] beskriver författarna hur man ökar effektiviteten i databaser\ldots'' utan snarare ``Reffersen beskriver~[Ref21] olika sätt att öka effektiviteten i databaser\ldots''.

För mer info om vilken info som behövs för olika typer av referenser, se avsnitt 8.5.3 i Dawson~\cite{dawson:projects-in-computing,dawson:projects-in-computing-old}. (För att referera till flera saker samtidigt (som nyss) skriver man flera BibTeX-nycklar i samma \verb|\cite|.) 
Där beskrivs också~\cite[sid 230]{dawson:projects-in-computing} hur man refererar till vad någon bara har sagt, utan att det finns publicerat (``personal communication''). Den typen av referenser är förstås svåra att kolla upp för läsaren, och typiskt ganska svaga, så det är viktigt att det framgår hur ``prominent'' personen som sagt saker är, hur tungt dennas ord ska vägas.

När ni använder referenser som finns i DiVA, utgå från de Bib\TeX-data ni får genom att klicka ``Exportera'' ovanför titeln och välj BibTex, men \emph{dubbelkolla} att resultatet blir bra! Specifikt för tidigare Självständiga arbeten i IT: ersätt bibtex-typen \verb|@misc| med \verb|@techreport| och fältet \verb|series| med \verb|type|. Lägg gärna till själva diva-länken (``Permanent länk'' under ``Länk till poster'') som \verb|url|-fält. Se exempel~\cite{Brane973772,Alstergren1439802} i \texttt{refs.bib}.

Använd inte URLer till universitetets ezproxy-tjänst (t.ex.~\url{www-sis-se.ezproxy.its.uu.se}), eftersom den bara kan användas av personer med UU-konto. Använd en direkt länk eller en DOI-länk istället.

Observera att nyhetsartiklar (tidningsartiklar och motsvarande) nästan alltid har ett publiceringsdatum, som ska visas (t.ex. i \verb|note|-fältet), och oftast också en artikelförfattare.

Använd inte direktcitat, såvida inte den exakta formuleringen är viktig.  Skriv hellre ett referat av vad någon sagt. (Se Dawson~\cite{dawson:projects-in-computing,dawson:projects-in-computing-old}.)

Om referenslistan huvudsakligen innehåller referenser till ``mer info'' av typen 
\url{www.wordpress.org}, \url{www.w3c.org}, \url{developer.android.com}\ldots men få referenser som stöder resonemang, motivation, argument etc (jfr Workshoparna), är det antagligen ett tecken på att det finns få resonemang, motiveringar och argument som behöver stödjas. Då behöver man med största sannolikhet resonera, motivera och argumentera mera!

Även om en referens har en URL till själva texten är det inte nödvändigtvis en webbreferens, utan ibland en artikel/bok el.dyl som råkar vara tllgänglig på nätet. Den ska då beskrivas som artikel/bok/el.dyl (men förstås gärna med URLen) så att man kan göra en preliminär värdering av referensen redan när man läser referenslistan.

Använd gärna Wikipedia eller en populärvetenskaplig artikel som en \emph{start} för att läsa om något, men ansträng er att hitta stabilare referenser när ni skriver om detta nägot. Kolla vilka referenser artikeln använder, läs och använd (åtminstone någon av) dem istället.

Titta på er referenslista och reflektera över vilka typer av referenser ni använt. Är det bara onlinereferenser? Är det bara ``läs-mera-här''-referenser? Eller har ni använt referenser på ett bra sätt för att underbygga era val, resonemang, argument och påståenden?

Försök hitta författare och publiceringsdatum (år, månad) även för webbreferenser, och ange \textbf{alltid} när de lästes, eftersom de kan uppdateras när som helst. Ett exempel är~\cite{berners-lee:cool-uris} (se \texttt{refs.bib}).



\section{Formler, figurer, bilder, kod}
\label{sec:forml-figur-bild}

Formler, figurer och ekvationer måste beskrivas.  Det betyder t.ex. att varje symbol måste vara förklarad i texten.

Låt figurer och tabeller (``floats'') hamna där La\TeX{} tycker att de ska, och justera bara placeringen i slutversionen och om det verkligen behövs, t.ex. för att de flyter iväg flera sidor.

Figurtexter (captions) ska beskriva vad vi ser i figuren, inte bara vad det är för slags figur. Att skriva ``Systemstruktur'' eller ``The structure of our system'' för en bild av systemstrukturen är inte tillräckligt. Hjälp läsaren förstå genom att också (eller istället) beskriva innehållet, t.ex. ``De gröna cirklarna representerar användare, och komponenter med skuggad bakgrund är externa. Indata kommer från vänster, och utdata levereras till höger''. Det räcker alltså \emph{inte} att beskriva figuren i löptexten -- men naturligtvis ska den beskrivas där också.

\label{figurers-namn}
I engelsk text skriver man ``Figure 3'', inte ``figure 3'', eftersom det fungerar som ett namn på figuren (och motsvarande för Table, Section, Appendix osv).

Alla figurer och bilder som inte är era egna måste ha referenser till källan.

Rapportmallen är inställd så att figurer presenteras med en linje över, en under, och en mellan figurtexten och själva bilden. För andra presentationer, se macrot \verb|\floatstyle| (googla) -- exempelvis ger \verb|\floatstyle{boxed}| istället en ram runt figuren.

Om ni inkluderar kodsnuttar, se till att de är relevanta och kommenterade, så att man förstår.  Alternativt, för korta snuttar: ge motsvarande förklaring i texten.
Använd vettigt latex-bibliotek för kod, t.ex. \texttt{listings}.

\section{Språk, grammatik, stil}
\label{sec:sprak-och-grammatik}

\begin{itemize}
\item    Det är OK att skriva ``Vi''!

\item    \textbf{Inte alla läsare är män}.  Skriv därför inte ``han'', ``hans'', ``denne'' etc.  Använd könsneutrala pronomen eller ord som ``vederbörande'', ``användaren'' etc. 

\item På engelska, undvik det informella \emph{you} som översättning av ``man'' -- undvik också \emph{one} som kan upplevas som alltför formellt (drottningen uttrycker sig så). Skriv ut vem som menas, t.ex. ``the user'' el.dyl., eller använd ``they'' (även i singular).

\item    \textbf{Undvik talspråk} ``så'', ``två stycken saker'', ``ifrån'', ``utav'', ``vart'', ``kommer göra/vara'' (istället för ``kommer att göra/vara'', \ldots \textbf{Kolla på Wikipedia-sidan} ``Vanliga språkfel''~\cite{wp:sprakfel}.

\item    Undvik värderande uttryck som enkelt, uppenbart.

\item Undvik att sluta meningar med en preposition (t.ex. \emph{med} eller \emph{with}).

\item    Semikolon är \textbf{inte} en variant av kolon eller komma; semikolon kan endast an\-vän\-das där ni normalt sett skulle använt punkt, men vill fortsätta på samma mening. För att undvika problem, undvik semikolon helt.

\item    Skriv inte meningar som börjar med ``Detta på grund av'' eller ``Detta eftersom\ldots' -- det blir ofta inte fullständiga meningar och det är ofta inte klart vad ``detta'' syftar på.

\item    Använd inte framtid (futurum); skriv rapporten i nu- eller dåtid och var konsekventa (Vi gör\ldots eller Vi har gjort\ldots, inte Vi kommer att göra\ldots). 

\item Var noggranna med valet av dåtid eller nutid: skriv inte i nutid om saker som inte nödvändigtvis gäller när texten läses (t.ex. om fem år), om det inte är tydligt \emph{när} det gäller. Gör det tydligt när det gäller.

\item Om ni skriver på engelska kan verktyget Grammarly (\url{www.grammarly.com}) vara användbart för att kolla grammatik, för svensk text kan man prova att klippa-och-klistra texten till ett dokument i Word, som har inbyggd koll av grammatik och stavning.

\label{app:definiera-innan-anv}
\item    \textbf{Förklara begrepp innan ni använder dem}, hänvisa inte \emph{bara} läsaren till ett senare avsnitt (men ni kan naturligtvis också hänvisa till mer detaljerade förklaringar som kommer senare i texten).  Första gången ett begrepp nämns måste alltså åt\-min\-sto\-ne en kort förklaring finnas.

\item Undvik helst att lägga in en ordlista/glossary i början av rapporten. När man läser den kommer begreppen utan kontext och det kan vara svårt att förstå. Ni måste ändå förklara begreppen när de behöver förklaras (se punkten ovan).

\item    När ni introducerar nya koncept (sådant ni inte har diskuterat tidigare), gör inte det ``i förbifarten'', utan se till att ni \textbf{förklarar ordentligt}.  Alltså: ``Vi använder X (ett häftigt nytt programmeringsspråk) för att göra Y'' fungerar inte.  Beskriv först konceptet ni använder, och använd det sedan.  Typ ``X är ett viktigt nytt programmeringsspråk.  Vi använder X för att göra Y.''

\item    \textbf{Var konsekventa} med hur ni skriver förkortningar och begrepp (c++ eller C++, android och Android t.ex.) Tumregel: namn skrivs med inledande stor bokstav (Android, inte android), förkortningar med stora bokstäver (XML, inte Xml).

\item    Använd inte olika synonymer för det ni har utvecklat (tjänsten/projektet/systemet), utan bestäm er för vad ni kallar det ni har gjort.

\item    Det kan vara bra att kursivera nya begrepp första gången de används, men normalt bör man inte kursivera \emph{alla} förekomster.

\item    Efter uttryck som ``för det första\ldots'', ``one alternative is\ldots'' måste följa ``för det andra\ldots'' ``another alternative'' (inte ``slutligen'', ``dels'', ``another \underline{option}'' eller något annat).  Tänk också på ``firstly \ldots secondly'' resp. ``first \ldots second'', inte ``first \ldots secondly'' eller något annat.

\item    Var försiktig med uttryck som ``this approach'', ``detta system'', etc. och kontrollera att det är uppenbart vad detta/this refererar till. Be någon icke-gruppmedlem läsa och kolla!

\item Undvik kontraktioner (sammandragningar) på engelska: skriv ``do not'' istället för ``don't'', och ``is not'' istället för ``isn't'' osv. Var extra försiktig med ``it's'' och ``its'' och ``it is'', och välj rätt.

  
\item På engelska, se upp med skillnaden mellan plural-s och genitiv-s, hur man an\-vän\-der dem tillsammans, och hur det blir när grundordet slutar på s.

\item På engelska måste ni använda korrekta verbformer beroende på om subjektet är en eller flera saker (``it has'' men ``they have'').

\item Undvik att skriva så här långa punktlistor, och punktlistor med så här mycket text.
\end{itemize}


%%% Local Variables:
%%% mode: latex
%%% TeX-master: "rapport-mall"
%%% End:


% Om ni har ett index
\makeatletter
\renewenvironment{theindex}
               {\if@twocolumn
                  \@restonecolfalse
                \else
                  \@restonecoltrue
                \fi
                \twocolumn[\section{\indexname}]%
                \@mkboth{\MakeUppercase\indexname}%
                        {\MakeUppercase\indexname}%
                \thispagestyle{plain}\parindent\z@
                \parskip\z@ \@plus .3\p@\relax
                \columnseprule \z@
                \columnsep 35\p@
                \let\item\@idxitem}
               {\if@restonecol\onecolumn\else\clearpage\fi}
\makeatother
\printindex
\end{document}
