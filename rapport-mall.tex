% MUST use a4paper option
% MAY use twoside, smaller font, and other class
\documentclass[a4paper,12pt]{article}
% Use UTF-8 encoding in input files
\usepackage[utf8]{inputenc}

% Om ni skriver på svenska, använd denna rad:
\usepackage[english,swedish]{babel}
% If you are writing in English, use the following line INSTEAD of the previous (note order of parameters):
% \usepackage[swedish,english]{babel}

% Use the template for thesis reports
\usepackage{UppsalaExjobb}

% Designval: per default används styckesindrag, men ibland blir det snyggare/mer lättläst med tomrad mellan stycken. Det åstadkoms av de följande raderna.
% Tycker ni om styckesindrag mera, kommentera bort nästa två rader.
\parskip=0.8em
\parindent=0mm

% Designval: vill ni ha en box runt figurer istället för strecken som är default, av-kommentera raden nedan. Obs att både \floatstyle och \restylefloat behövs.
%\floatstyle{boxed} \restylefloat{figure}

%%%% OBS:
% När ni inte längre behöver instruktionerna kan ni ta bort allt mellan \begin{instructions} och \end{instructions} (inklusive de raderna),
% och mellan \ifinstructions och matchande \fi (inklusive de raderna).

\begin{document}
% För att ställa in parametrar till IEEEtranS/IEEEtranSA behöver detta ligga här (före första \cite).
% Se se IEEEtran/bibtex/IEEEtran_bst_HOWTO.pdf, avsnitt VII, eller sista biten av IEEEtran/bibtex/IEEEexample.bib.
%%%% OBS: här ställer ni t.ex. in hur URLer ska beskrivas.
\bstctlcite{rapport:BSTcontrol}

% Set title, and subtitle if you have one
\title{Rapportmall för självständigt arbete} % och uppsatsmetodik
% Use this if you have a subtitle
%\subtitle{beskrivande men gärna lockande}
\subtitle{version sommar 2021}

% Set author names, separated by "\\ " (don't forget the space, or use newline)
% List authors alphabetically by LAST NAME (unless someone did significantly more/less, which should not be the case)
% For drafts, include your email addresses to make it easier to send peer reviews
\author{Sofia Cassel \\ Björn Victor (bjorn.victor@it.uu.se)}

% Visa datum på svenska på förstasidan, även om ni skriver på engelska!
\date{\begin{otherlanguage}{swedish}  %\foreignlanguage doesn't seem to affect \today?
\today
\end{otherlanguage}}

% Använd detta om året för rapporten inte är innevarande år
%\year=2018

% Ange handledare, ämnesgranskare, examinator om dessa finns
% Extern handledare: t.ex på företag ni arbetat med?
\exthandledare{Hand Ledare, Firma Ment AB}
% Intern(a) handledare, i bokstavsordning på efternamnet
\handledare{NN och Björn Victor}

% This creates the title page
\maketitle

% Change to frontmatter style (e.g. roman page numbers)
\frontmatter

%%%% OBS: Läs också källkoden till alla instr-X.tex.
%%%% Tips: ni kan använda separata filer för de olika delarna i er rapport på motsvarande sätt,
%%%% men använd inte samma filnamn!

\begin{abstract}
\input{instr-abstract}
\end{abstract}

\begin{sammanfattning}
Sammanfattning, alltid på svenska. Se till att det står samma saker i den svenska sammanfattningen och det engelska abstractet.
\begin{enumerate}
\item Vad är problemet, ämnet?
\item Hur angreps/löstes problemet?
\item Vad är resultaten, hur väl löstes problemet?
\item Hur bra blev resultaten, hur användbara är de?
\end{enumerate}

Ca 10-20 rader. Använd inte referenser; ej heller formler om det går att undvika.

Sammanfattningen ska vara förståelig utan att läsa resten av rapporten, och resten av rapporten ska kunna läsas utan att läsa sammanfattningen. Det är helt OK att återanvända text från introduktionen.


\end{sammanfattning}

% Innehållsförteckningen här.
\tableofcontents

% Här kan man också ha \listoffigures, \listoftables

\cleardoublepage

%%%%%%%%%%%%%%%% Ta bort allt mellan här och \mainmatter (inkl \newpage) (men inte \mainmatter) i slutversionen
\section*{Hur ni använder detta malldokument}
Titta i källdokumentet för diverse inställningar för författare, titel, etc. Läs också käll\-doku\-men\-ten för instruktionerna i filerna \verb|instr-X.tex| för olika värden på \verb|X|.

\emph{OBSERVERA} att de ``fasta fält'' som blir på svenska (trots att ni ställt in engelska med \texttt{babel}), som Examinator, Handledare, datum på framsidan osv, \emph{ska} vara på svenska oavsett språk i rapporten. Abstract ska alltid vara på engelska, medan Sammanfattning alltid ska vara på svenska.

I flera appendix finns mer info som inte gäller rapportstrukturen.

För att slippa få med instruktionerna för rapportstrukturen i era inlämningar, ta bort \verb|\input{instr-X}| för alla värden av \verb|X|
i källdokumentet.

\textbf{Tips:} ni kan använda separata filer för de olika delarna i er rapport på motsvarande sätt, men använd inte samma filnamn!

\subsection*{Generellt}
Varje numrerat avsnitt ska finnas med i er slutrapport, om inget annat anges.  
Välj rubrik på svenska eller engelska beroende på ert valda rapportspråk.

Om ni skriver på engelska ska titeln skrivas med första bokstaven i varje ord versal, utom ``småord''. Exempel: \emph{A Really Interesting Project on the Fundamentals of Shoes}\footnote{Se t.ex.~\url{https://en.wikipedia.org/wiki/Capitalization\#Titles}}.  (Detta gäller även titlar i referenser på engelska.)
Rubrikerna i texten kan skrivas på detta sätt eller som på svenska (stor första bokstav i meningen), men \emph{var konsekventa}.

Glöm inte att läsa kurslitteraturen~\cite{dawson:projects-in-computing,dawson:projects-in-computing-old}.

% \subsection*{Uppdateringar av detta dokument}
% \begin{description}
% \item[2016-05-16]\mbox{}\\

% \end{description}


\section*{Att göra}
En sektion som beskriver läget för rapporten kan vara användbart i ``veckans inlämning'' för att underlätta feedbacken.

För att hantera ``att-göra-listor'' i rapporten kan La\TeX-paketet \verb|todonotes| kanske vara användbart. Se \url{http://ctan.org/pkg/todonotes} för mer info.



\newpage
%%%%%%%%%%%%%%%% OBS! Ta bort allt mellan \mainmatter och här (inkl \newpage) i slutversionen

% Change to main matter style (arabic page numbers, reset page numbers)
\mainmatter

% Here comes the text of the report.

\section{Introduktion eller Inledning / Introduction}
\label{sec:introduktion}
\input{instr-introduktion}

\paragraph{Tillkännagivande eller Tack / Acknowledgement}
\input{instr-acknowledgement}

\paragraph{Redovisning av arbetsfördelning / Declaration of division of labor}
\input{instr-arbete}


\section{Bakgrund / Background}
\label{sec:bakgrund}
\input{instr-bakgrund}

\section{Syfte, mål, och motivation / Purpose, aims, and motivation}\label{sec:syfte}
Här beskriver ni i princip er problemformulering.  I detta avsnitt ska framgå syfte, mål, och motivation med projektet. 
Dessa behöver dock \emph{inte} vara separata underrubriker.

\paragraph{Syfte.} Vart strävar projektet? vad är det övergripande målet, nyttan, effekterna av projektet?  (t.ex. bättre koll på kosthållning, enklare planering av studier\ldots)
\paragraph{Mål.} Vad ska konkret levereras/utföras av projektet, för att ta oss närmare syftet?
\paragraph{Motivation.}  Varför är projektet viktigt?  Vilka är det viktigt för, vilka externa intressenter finns?  Hur stort är problemet, vad är följden av att det inte är löst, hur bra vore det att lösa?  Vilken ``lucka'' i området täcker ni?
Varför är er lösning bättre/annorlunda än andras?

Se till att ni i detta avsnitt övertygar läsaren om att problemet finns, att det inte är löst, och att det är viktigt att lösa. Ju starkare argumentation och motivation (med källor) dess bättre.
\begin{itemize}
\item Visa att det finns ett problem.
\item Visa att problemet är viktigt att lösa, att det behöver lösas.
\item Visa att problemet inte redan är löst.
\end{itemize}

I det här avsnittet kan ni också börja beskriva etiska aspekter och hållbarhetsaspekter, men det finns förstås flera naturliga ställen att ta upp dem (\emph{till exempel} sektionerna~\ref{sec:metod}, \ref{sec:krav}, \ref{sec:resultat} och~\ref{sec:slutsatser}, men kanske redan i sektion~\ref{sec:introduktion} eller \ref{sec:bakgrund}).

Det är helt OK (och bra!) att också beskriva negativa/kritiska aspekter av ert projekt och arbete, inte bara positiva/goda. 

\emph{Använd kursmaterialet} för att få stöd att utveckla etiska och hållbarhetsaspekter.
Tänk t.ex.~på stödfrågorna för att tänka på etiska aspekter:
\begin{itemize}
\item Vilka är \emph{direkta} intressenter och hur påverkas de? (användare, företag, kunder)
\begin{itemize}
\item  Vad krävs för att kunna använda er lösning? (kunskap, förmågor, resurser)
\item  Vilka exkluderar ni?
\item  Vad underlättar ni och vad gör ni svårare?
\item  I vilka sammanhang kan er lösning användas och inte användas?
\end{itemize}
\item  Vilka är \emph{indirekta} intressenter och hur påverkas de? (familj, samhälle, konkurrenter)
\item  Kan tekniken användas för ``fel'' syften?
\item  Hur ser ett samhälle ut där er lösning används i stor skala?
\end{itemize}

För hållbarhetsfrågor, prova gärna ``Futures Thinking'' och ``Systems Thinking'' (se länkar i kursmaterialet).


\subsection{Avgränsningar / Delimitations}
\input{instr-avgransningar}

\section{Relaterat arbete / Related work}
\input{instr-relaterat}

\section{Metod/Tillvägagångssätt/Tekniker eller Method/Approach/Techniques}
\label{sec:metod}
Här beskriver ni vilka metoder/verktyg/tekniker/approacher ni använt för att lösa problemet / besvara frågeställningen.  Vilka metoder har ni konkret använt för att lösa problemet/bygga systemet?  Vilka tekniker/verktyg använde ni?

Observera att det inte är samma sak som att beskriva \emph{hur} ni använde teknikerna/verktygen: det kommer i Del X, implementationsdelen (se avsnitt~\ref{sec:del-x}).

Glöm inte att \emph{motivera} era val av metoder. Finns det flera rimliga alternativ? Beskriv varför ni inte valt dem (t.ex.~varför er valda metod är bättre).
Visa att det är rimligt att använda just detta tillvägagångssätt.
Det gäller även i det fall det är givet på förhand vilken teknik ni ska använda (t.ex. vilket programspråk) för att det ska passa i ett sammanhang eller existerande system (t.ex. ett som ny bygger vidare på). 

Detta avsnitt ska \emph{inte} innehålla information om hur gruppen organiserat arbetet (github, trello\ldots) \emph{om} det inte är relevant för resultatet (och det är det oftast inte).

Använd tydliga underrubriker, t.ex. ``Ramverk för webbplatsen'' snarare än ``Wordpress'', eller ``Databashanterare'' snarare än ``MongoDB''.

%%% Local Variables:
%%% mode: latex
%%% TeX-master: "rapport-mall"
%%% End:


\section{Systemstruktur / System structure}
\input{instr-systemstruktur}

\section{Krav och utvärderingsmetoder / Requirements and evaluation methods}
\label{sec:krav}
För de olika funktionaliteterna (och/eller motsv) i ert system, hur ska ni avgöra om de är tillräckligt bra utförda/implementerade? Var går gränsen för ``tillräckligt bra''? (Eller när är de ``för dåliga''?)

Skilj på funktionalitet (vad ska systemet kunna göra) och krav (hur bra ska systemet vara). Själva funktionaliteterna har ni redan beskrivit i systemstrukturen eller huvuddelen nedan. (Har ni krav på saker ni beskriver först i huvuddelen kan ni lägga det här avsnittet efter huvuddelen.)

Skriv tydliga krav \emph{som går att utvärdera}.  (Hur snabbt? Hur många användare? Hur strömsnålt? eller vad som är relevant).

Beskriv hur utvärderingen ska gå till (automatiserade belastningstester, mätningar, en\-käter, fokusgrupper\ldots).
Beskriv hur externa intressenter involveras i utvärderingen.


\section{DEL x: Implementation av XYZ}
\label{sec:del-x}
\label{sec:delX}
Mellan introduktion och avslutning finns ett eller sannolikt \emph{flera} avsnitt (``huvuddelen'') som innehåller själva bidraget eller implementationen.
Ni får själva välja passande rubriker (INTE ``Huvuddel'' eller ``Bidrag'').  Rubrikerna i huvuddelen ska tillsammans med titeln ge en idé om vad som berättas, en ``berättelse''. (Exempel: ``Algoritm för automatisk igenkänning av stora fötter'', ``Design av databasen för användardata'', ``Optimering av minnesanvändning'', ``Implementation av djup\-in\-lär\-nings\-sys\-te\-met'' etc.)

Här kan ni beskriva implementationen, hur systemet används, etc.

Beskriv gärna felhantering och riskanalys: vad kan gå fel när systemet kör/används, vad kan bli följden, och hur hanteras detta?

\section{DEL x+1: Algoritm för igenkänning av stora fötter}
Se avsnitt~\ref{sec:delX}.
\section{DEL x+2: Optimering av minnesanvändning}
Se avsnitt~\ref{sec:delX}.

\ldots


\section{Utvärderingsresultat / Evaluation results}
\input{instr-utv-resultat}

\section{Resultat och diskussion / Results and discussion}
\label{sec:resultat}
\input{instr-resultat}

\section{Slutsatser / Conclusions}
\label{sec:slutsatser}
Här sammanfattar ni och upprepar ert bidrag (resultaten av ert projekt) och förklarar dess vikt och användning.  Vad var viktigt/nytt/intressant?  (INTE i termer av vad ni lärde er, utan för den som läser rapporten, funderar på att göra ett liknande system, vidareutveckla ert system, etc.)


\section{Framtida arbete / Future work}
\input{instr-framtida-arbete}

%%%% Referenser - SE OCKÅ APPENDIX

% Use one of these:
%   IEEEtranS gives numbered references like [42] sorted by author,
%   IEEEtranSA gives ``alpha''-style references like [Lam81] (also sorted by author)
%\bibliographystyle{IEEEtranS}
\bibliographystyle{IEEEtranSA}

% Here comes the bibliography/references.
% För att göra inställningar för IEEEtranS/SA kan man använda ett speciellt bibtex-entry @IEEEtranBSTCTL,
% se IEEEtran/bibtex/IEEEtran_bst_HOWTO.pdf, avsnitt VII, eller sista biten av IEEEtran/bibtex/IEEEexample.bib.
\bibliography{bibconfig,refs}
%\bibliography{refs}

\newpage
\appendix %%%% markerar att resten är appendixar
%%%% I er egen version, ta bort allt nedan (utom \end{document})
%%%% Här finns både instruktioner och tips - läs hela!

\section{Hur man gör appendix}
Appendixar kan vara bra för bilagor som enkätundersökningar, större kodavsnitt, etc. 

Appendix läggs efter referenslistan, och ska börja på en ny sida. Använd \verb|\newpage| för att göra ett sidbrott där resten av nuvarande sida är tom. Skriv sen \verb|\appendix| för att markera att resten är appendix, och 
 använd sen vanliga \verb|\section{}| för varje appendix, som kommer att ``numreras'' A, B, C osv.

\section{Några tips för La\TeX-användning}

Ett enkelt sätt att använda/\textbf{installera} LaTeX för MacOS är TexShop\footnote{\url{http://pages.uoregon.edu/koch/texshop}}.

För \textbf{samarbete} när man skriver La\TeX-dokument använder somliga Overleaf (\url{https://www.overleaf.com/}, tips på liknande system är välkomna), men det funkar också att använda git och vanliga texteditorer (t.ex Emacs). I det fallet är det smart att dela upp dokumentet i flera (t.ex. ett per kapitel) som sen inkluderas med \verb|\input{kapitel}|. Ett tips är också att slå på radbrytning i texteditorn, så att konflikter vid incheckningar hanteras per kort rad istället för per jättelång rad.

\textbf{Läs också i Wikibooks} (\url{http://en.wikibooks.org/wiki/LaTeX}), \textbf{missa inte} Appendix om ``Sample LaTeX documents'' (men använd alltid rapportmallen som bas).

\textbf{Citat-tecken} skriver man med \verb|``foo''| (dvs två bakåtfnuttar före, och två vanliga fnuttar efter). LaTeX gör så att det blir snyggt: ``foo''.

När man skriver på svenska behöver man ibland ``visa'' var ord (speciellt såna med med åäö) kan \textbf{avstavas} genom att använda \verb|\-| (liknande \textit{soft hyphen}): ämnesöversiktsintroduktion avstavas med några sådana instuckna på rätt ställen istället som ämnes\-över\-sikts\-intro\-duk\-tion

\begin{verbatim}
ämnes\-över\-sikts\-intro\-duk\-tion
\end{verbatim}

För att formattera \textbf{URLer} bättre (så att t.ex. radbrytning blir snyggare), skriv t.ex. \verb|\url{http://www.it.uu.se/research/group/concurrency}| i texten eller referensen.

För att \textbf{referera} till avsnitt, figurer, tabeller etc, använd \verb|\label{markör}| för att ``sätta ett märke'' i text eller figur, och \verb|\ref{markör}| för att referera till den, t.ex.
\begin{verbatim}
\section{Motivation}
\label{sec:motivation}
\end{verbatim}

följt av
\begin{verbatim}
Som vi nämnt i avsnitt~\ref{sec.motivation}...
\end{verbatim}
eller på engelska (notera ``Section'' med stor inledande bokstav)
\begin{verbatim}
As already mentioned in Section~\ref{sec.motivation}...
\end{verbatim}


För att få referenser att inte hamna först efter ett \textbf{radbrott}, använd icke-brytande space \verb|såhär~\cite{fin-bok}|, där tilde-tecknet \verb|~| alltså gör ett obrytbart space. Detta är i princip också alltid rätt att använda före siffror (och i stora tal på engelska, t.ex. \verb|100~000| för 100~000), och förstås också före \verb|\ref{fig}|.

Använd \emph{aldrig} dubbel-backslash \verb|\\| för att få avbrott mellan stycken. Använd alltid dubbel ny rad för detta. Använd bara \verb|\\| i tabeller o.dyl.

För att göra ett \textbf{sidbrott} där resten av sidan blir tom, använd \verb|\newpage|, använd inte \verb|\pagebreak|. Det senare är till för att finjustera var latex gör ett automatiskt sidbrott, inte för att avsluta en halvfull sida.

\subsection{Bib\TeX-tips}

För att hantera bibliografi (\textbf{referenser}) på ett smidigt sätt, använd BibTeX! (se \url{http://en.wikibooks.org/wiki/LaTeX/Bibliography_Management#BibTeX} och nedan om referenser.)

För att se till att BibTeX inte gör namn, förkortningar etc till lowercase, använd \verb|{}| och skriv typ
\begin{verbatim}
title = {The {DSP} of {N}ewton applied to {iOS}}
\end{verbatim}

Skriv alltid månader för publikation med de inbyggda förkortningarna, typ:
\begin{verbatim}
month = jun
\end{verbatim}
istället för \verb|{jun}| eller \verb|"jun"| eller \verb|"June"| eller \verb|"Juni"|. Då kan nämligen bibliographystyle styra hur det förkortas etc.

Ett verktyg för att hantera BibTeX-filer i MacOS är BibDesk\footnote{\url{http://bibdesk.sourceforge.net/}}.

\section{Referenser}
\label{sec:referenser}

Se också kap 8.5 i Dawson~\cite{dawson:projects-in-computing}.

Det finns åtminstone tre syften med utformningen av referenserna och referenslistan.
\begin{enumerate}
\item Man ska hitta referensen (från texten) i referenslistan.
\item Man ska förstå vad som refereras (vilken typ av referens det är) så att man kan värdera den.
\item Man ska kunna hitta referensen i verkligheten.
\end{enumerate}

Använd numeriska referenser (IEEE-stil~[42]) eller nyckelordsbaserad~[Lam86], inte fotnotstil. Referenserna sorteras alfabetiskt efter författare/motsv i referenslistan. I LaTeX, använd \verb|\bibliographystyle{IEEEtranS}| eller \verb|{IEEEtranSA}| (eller liknande), se rapportmallen. \textbf{Börja} med att använda \verb|{IEEEtranSA}| som tydligare visar när viss info saknas i bibtex-entries (t.ex. år och författare).

För att göra inställningar för \verb|\bibliographystyle{IEEEtranS/SA}| kan man använda ett speciellt bibtex-entry \texttt{@IEEEtranBSTCTL}, se \texttt{IEEEtran\_bst\_HOWTO.pdf} i directoryt \texttt{IEEEtran/bibtex}, avsnitt VII, eller sista biten av \texttt{IEEEexample.bib} i samma directory.

Referenserna skrivs i direkt anknytning till det som föranleder referensen (t.ex. ett påstående eller resultat), före eventuellt skiljetecken, och med ett fast mellanslag till föregående ord. I La\TeX, \verb|skriv~\cite{lam86}| för att få en ``non-breaking space''. Se också rapportmallen, och sista stycket på sid 211 i Dawson~\cite{dawson:projects-in-computing}. 

Man ska alltså \emph{inte} skriva referenserna efter ett längre stycke (som vissa verkar lära sig att göra, någonstans). Det gör det oftast otydligt vad som egentligen är hämtat från, eller styrks, av referenserna. I vissa fall kan man vilka göra en kort sammanfattning av vad en författare skriver i en artikel el.dyl., men att bara lägga på en referens sist i stycket är inte tillräckligt tydligt. Det är mycket bättre och tydligare att inleda stycket med att skriva något i stil med ``Lisa Lagom beskriver\verb|~\cite{lagom-bok}| hur X beror av Y och i sin analys visar hon i detalj hur sambandet ser ut\ldots''.

När man refererar till ``tjocka'' saker som böcker är det lämpligt att ange sidnummer 
(som \verb|\cite[sid 211-214]{dawson:projects-in-computing}|) som blir \cite[sid 211-214]{dawson:projects-in-computing}, men för ``tunnare'' saker behöver man bara göra det för att speciellt peka ut om man t.ex. menar en viss del av referensen (kanske den tar upp tre olika sätt att göra X och man vill peka på det 3:e, inte de första två).

För mer info om vilken info som behövs för olika typer av referenser, se avsnitt 8.5.3 i Dawson~\cite{dawson:projects-in-computing,dawson:projects-in-computing-old}. (För att referera till flera saker samtidigt (som nyss) skriver man flera BibTeX-nycklar i samma \verb|\cite|.) 
Där beskrivs också~\cite[sid 230]{dawson:projects-in-computing} hur man refererar till vad någon bara har sagt, utan att det finns publicerat (``personal communication''). Den typen av referenser är förstås svåra att kolla upp för läsaren, och typiskt ganska svaga, så det är viktigt att det framgår hur ``prominent'' personen som sagt saker är, hur tungt dennas ord ska vägas.

Observera att nyhetsartiklar (tidningsartiklar och motsvarande) nästan alltid har ett publiceringsdatum, som ska visas (t.ex. i \verb|note|-fältet), och oftast också en artikelförfattare.

Använd inte direktcitat, såvida inte den exakta formuleringen är viktig.  Skriv hellre ett referat av vad någon sagt. (Se Dawson~\cite{dawson:projects-in-computing,dawson:projects-in-computing-old}.)

Om referenslistan huvudsakligen innehåller referenser till ``mer info'' av typen 
\url{www.wordpress.org}, \url{www.w3c.org}, \url{developer.android.com}\ldots men få referenser som stöder resonemang, motivation, argument etc (jfr Workshoparna), är det antagligen ett tecken på att det finns få resonemang, motiveringar och argument som behöver stödjas. Då behöver man med största sannolikhet resonera, motivera och argumentera mera!

Även om en referens har en URL till själva texten är det inte nödvändigtvis en webbreferens, utan ibland en artikel/bok el.dyl som råkar vara tllgänglig på nätet. Den ska då beskrivas som artikel/bok/el.dyl (men förstås gärna med URLen) så att man kan göra en preliminär värdering av referensen redan när man läser referenslistan.

Använd gärna Wikipedia som en start för att läsa om något, men ansträng er att hitta stabilare referenser när ni skriver om detta nägot. Kolla vilka referenser Wikipedia-artikeln använder, läs och använd (åtminstone någon av) dem istället.

Titta på er referenslista och reflektera över vilka typer av referenser ni använt. Är det bara onlinereferenser? Är det bara ``läs-mera-här''-referenser? Eller har ni använt referenser på ett bra sätt för att underbygga era val, resonemang, argument och påståenden?

Försök hitta författare och publiceringsdatum (år, månad) även för webbreferenser, och ange \textbf{alltid} när de lästes, eftersom de kan uppdateras när som helst. Ett exempel är~\cite{berners-lee:cool-uris} (se \texttt{refs.bib}).



\section{Formler, figurer, bilder, kod}
\label{sec:forml-figur-bild}

Formler, figurer och ekvationer måste beskrivas.  Det betyder t.ex. att varje symbol måste vara förklarad i texten.

Låt figurer och tabeller (``floats'') hamna där La\TeX{} tycker att de ska, och justera bara placeringen i slutversionen och om det verkligen behövs, t.ex. för att de flyter iväg flera sidor.

I engelsk text skriver man ``Figure 3'', inte ``figure 3'', eftersom det fungerar som ett namn på figuren (och motsvarande för Table, Section osv).

Alla figurer och bilder som inte är era egna måste ha referenser till källan.

Rapportmallen är inställd så att figurer presenteras med en linje över, en under, och en mellan figurtexten och själva bilden. För andra presentationer, se macrot \verb|\floatstyle| (googla) -- exempelvis ger \verb|\floatstyle{boxed}| istället en ram runt figuren.

Om ni inkluderar kodsnuttar, se till att de är relevanta och kommenterade, så att man förstår.  Alternativt, för korta snuttar: ge motsvarande förklaring i texten.
Använd vettigt latex-bibliotek för kod, t.ex. \texttt{listings}.

\section{Språk, grammatik, stil}
\label{sec:sprak-och-grammatik}

\begin{itemize}
\item    Det är OK att skriva ``Vi''!

\item    \textbf{Inte alla läsare är män}.  Skriv därför inte ``han'', ``hans'', ``denne'' etc.  Använd könsneutrala pronomen eller ord som ``vederbörande'', ``användaren'' etc. 

\item På engelska, undvik det informella \emph{you} som översättning av ``man'' -- undvik också \emph{one} som kan upplevas som alltför formellt (drottningen uttrycker sig så). Skriv ut vem som menas, t.ex. ``the user'' el.dyl., eller använd ``they'' (även i singular).

\item    \textbf{Undvik talspråk} ``så'', ``två stycken saker'', ``ifrån'', ``utav'', ``vart'', ``kommer göra/vara'' (istället för ``kommer att göra/vara'', \ldots \textbf{Kolla på Wikipedia-sidan} ``Vanliga språkfel''~\cite{wp:sprakfel} (länk i vänsterkanten i SP).

\item    Undvik värderande uttryck som enkelt, uppenbart.

\item Undvik att sluta meningar med en preposition (t.ex. \emph{med} eller \emph{with}).

\item    Semikolon är \textbf{inte} en variant av kolon eller komma; semikolon kan endast an\-vän\-das där ni normalt sett skulle använt punkt, men vill fortsätta på samma mening. För att undvika problem, undvik semikolon helt.

\item    Skriv inte meningar som börjar med ``Detta på grund av'' eller ``Detta eftersom\ldots' -- det blir ofta inte fullständiga meningar och det är ofta inte klart vad ``detta'' syftar på.

\item    Använd inte framtid (futurum); skriv rapporten i nu- eller dåtid och var konsekventa (Vi gör\ldots eller Vi har gjort\ldots, inte Vi kommer att göra\ldots). 

\item Var noggranna med valet av dåtid eller nutid: skriv inte i nutid om saker som inte nödvändigtvis gäller när texten läses (t.ex. om fem år), om det inte är tydligt \emph{när} det gäller. Gör det tydligt när det gäller.

\item    \textbf{Förklara begrepp innan ni använder dem}, hänvisa inte \emph{bara} läsaren till ett senare avsnitt (men ni kan naturligtvis också hänvisa till mer detaljerade förklaringar som kommer senare i texten).  Första gången ett begrepp nämns måste alltså åt\-min\-sto\-ne en kort förklaring finnas.

\item Undvik helst att lägga in en ordlista/glossary i början av rapporten. När man läser den kommer begreppen utan kontext och det kan vara svårt att förstå. Det är bättre att förklara begreppen när de behöver förklaras (se ovan).

\item    När ni introducerar nya koncept (sådant ni inte har diskuterat tidigare), gör inte det ``i förbifarten'', utan se till att ni \textbf{förklarar ordentligt}.  Alltså: ``Vi använder X (ett häftigt nytt programmeringsspråk) för att göra Y'' fungerar inte.  Beskriv först konceptet ni använder, och använd det sedan.  Typ ``X är ett viktigt nytt programmeringsspråk.  Vi använder X för att göra Y.''

\item    \textbf{Var konsekventa} med hur ni skriver förkortningar och begrepp (c++ eller C++, android och Android t.ex.) Tumregel: namn skrivs med inledande stor bokstav (Android, inte android), förkortningar med stora bokstäver (XML, inte Xml).

\item    Använd inte olika synonymer för det ni har utvecklat (tjänsten/projektet/systemet), utan bestäm er för vad ni kallar det ni har gjort.

\item    Det kan vara bra att kursivera nya begrepp första gången de används, men normalt bör man inte kursivera \emph{alla} förekomster.

\item    Efter uttryck som ``för det första\ldots'', ``one alternative is\ldots'' måste följa ``för det andra\ldots'' ``another alternative'' (inte ``slutligen'', ``dels'', ``another \underline{option}'' eller något annat).  Tänk också på ``firstly \ldots secondly'' resp. ``first \ldots second'', inte ``first \ldots secondly'' eller något annat.

\item    Var försiktig med uttryck som ``this approach'', ``detta system'', etc. och kontrollera att det är uppenbart vad detta/this refererar till. Be någon icke-gruppmedlem läsa och kolla!

\item Undvik kontraktioner (sammandragningar) på engelska: skriv ``do not'' istället för ``don't'', och ``is not'' istället för ``isn't'' osv.

\item    De av er som skriver på engelska: ni MÅSTE använda korrekta verbformer beroende på om subjektet är en eller flera saker (``it has'' men ``they have'').

\item Undvik att skriva så här långa punktlistor, och punktlistor med så här mycket text.
\end{itemize}



\end{document}
